%%%%%%%%%%%%%%%%%%%%%%%%%%%%%%%%%%%%%%%%%
% University/School Laboratory Report
% LaTeX Template
% Version 3.1 (25/3/14)
%
% This template has been downloaded from:
% http://www.LaTeXTemplates.com
%
% Original author:
% Linux and Unix Users Group at Virginia Tech Wiki
% (https://vtluug.org/wiki/Example_LaTeX_chem_lab_report)
%
% License:
% CC BY-NC-SA 3.0 (http://creativecommons.org/licenses/by-nc-sa/3.0/)
%
%%%%%%%%%%%%%%%%%%%%%%%%%%%%%%%%%%%%%%%%%

\documentclass{article}

%\usepackage[version=3]{mhchem} % Package for chemical equation typesetting
%\usepackage{siunitx} % Provides the \SI{}{} and \si{} command for typesetting SI units
\usepackage{graphicx} % Required for the inclusion of images
\usepackage{natbib} % Required to change bibliography style to APA
\usepackage{amsmath} % Required for some math elements
\usepackage{listings}
\lstset{basicstyle=\ttfamily, breaklines=true}
\setlength\parindent{2pt} % Removes all indentation from paragraphs

%\renewcommand{\labelenumi}{\alph{enumi}.} % Make numbering in the enumerate environment by letter rather than number (e.g. section 6)

%\usepackage{times} % Uncomment to use the Times New Roman font


%----------------------------------------------------------------------------------------
%	DOCUMENT INFORMATION
%----------------------------------------------------------------------------------------

\title{Design of Foscam Webcam Firmware\\ UML for Embedded Systems} % Title

\author{Simone \textsc{Rossi}} % Author name


\date{\today} % Date for the report

\begin{document}

\maketitle % Insert the title, author and date

\begin{center}
\begin{tabular}{l r}
Date Performed: & \today \\ % Date the experiment was performed
Partners: & Simone Rossi \\ % Partner names
%& Mary Smith \\
%Instructor: & Professor Smith % Instructor/supervisor
\end{tabular}
\end{center}

% If you wish to include an abstract, uncomment the lines below
% \begin{abstract}
% Abstract text
% \end{abstract}

%----------------------------------------------------------------------------------------
%	SECTION 1
%----------------------------------------------------------------------------------------

\section{Assumptions}
\begin{itemize}
\item \textbf{LightSensor\_value\_rate}: the light sensor provides a valid data once
per second
\item \textbf{Camera\_frame\_rate}: the camera provides a valid frame every 1/25 sec (40 ms)
\item \textbf{Request\_notification}: each incoming Ethernet request is preceeded by a request
notification
\item \textbf{QR\_configuration}:  the QR code is the serial number of the camera.
During configuration, the serial read by the smartphone is compared with the internal serial number.
If equal, then the communication is configured
\item \textbf{Recording\_video}: If the camera has been configured, it always records video and checks
for motion objects.
\end{itemize}


\end{document}
